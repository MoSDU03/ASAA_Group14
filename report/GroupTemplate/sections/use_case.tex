\section{Use Case and Quality Attribute Scenarios}

\subsection{System Scope}

The can filling system operates on a production conveyor line. A can arrives at the fill station where sensors detect position (±2mm tolerance), a controller opens a valve, level sensors monitor fill progress (target: 330ml ±5ml), and the controller closes the valve when target is reached. The system logs all operations to a database for quality assurance.

\textbf{Explicit exclusions:} Sealing operations, quality inspection beyond fill level, routing mechanisms, multi-product configurations, and batch management are out of scope to maintain focus on architecture principles.

\subsection{Quality Attribute Scenarios}

Following Bass et al.~\cite{bass_swa}, we define three quality scenarios:

\textbf{QAS-P1 (Performance):}
\begin{itemize}
    \item \textit{Source:} Conveyor system
    \item \textit{Stimulus:} Can arrives at fill station
    \item \textit{Environment:} Normal operation (20°C, nominal flow)
    \item \textit{Artifact:} Fill controller
    \item \textit{Response:} Complete detection -> fill -> release cycle
    \item \textit{Measure:} 600ms ≤ cycle time ≤ 1500ms
\end{itemize}

\textbf{QAS-S1 (Safety):}
\begin{itemize}
    \item \textit{Source:} Level sensor
    \item \textit{Stimulus:} Sensor fault detected
    \item \textit{Environment:} Active filling operation
    \item \textit{Artifact:} Fault handler
    \item \textit{Response:} Emergency valve closure
    \item \textit{Measure:} Response time < 50ms
\end{itemize}

\textbf{QAS-R1 (Reliability):}
\begin{itemize}
    \item \textit{Source:} Internal monitoring
    \item \textit{Stimulus:} Sensor fault occurs
    \item \textit{Environment:} Runtime operation
    \item \textit{Artifact:} Sensor data collector
    \item \textit{Response:} Fault detected and logged
    \item \textit{Measure:} Detection latency < 200ms
\end{itemize}

\subsection{Requirements}

Table~\ref{tab:requirements} lists 15 requirements derived from the quality scenarios. Eight functional requirements (FR-01 to FR-08) specify system behavior: can detection, position validation, fill level control, valve operation, sensor polling, operation logging, timeout detection, and can release. Seven non-functional requirements (NFR-01 to NFR-07) specify quality constraints: cycle time (600-1500ms), maximum fill time (3000ms), fill tolerance (±5ml), position tolerance (±2mm), fault detection latency (<200ms), emergency response time (<50ms), and success rate (>99%).

\begin{table}[htbp]
\caption{Requirements Specification}
\label{tab:requirements}
\centering
\small
\begin{tabular}{|l|l|l|}
\hline
\textbf{ID} & \textbf{Type} & \textbf{Description} \\
\hline
FR-01 & Func & Detect can arrival \\
FR-02 & Func & Validate position (±2mm) \\
FR-03 & Func & Control fill level (330ml ±5ml) \\
FR-04 & Func & Open/close valve on command \\
FR-05 & Func & Poll sensors at 20Hz \\
FR-06 & Func & Log all operations \\
FR-07 & Func & Detect position timeout \\
FR-08 & Func & Release can after completion \\
\hline
NFR-01 & Perf & Cycle time: 600-1500ms \\
NFR-02 & Perf & Max fill time: 3000ms \\
NFR-03 & Perf & Fill tolerance: ±5ml \\
NFR-04 & Safety & Position tolerance: ±2mm \\
NFR-05 & Rel & Fault detection: <200ms \\
NFR-06 & Safety & Emergency response: <50ms \\
NFR-07 & Rel & Success rate: >99\% \\
\hline
\end{tabular}
\end{table}

Each requirement links to quality scenarios and is verified through UPPAAL properties (Section~\ref{sec:verification}) and empirical testing (Section~\ref{sec:evaluation}).
