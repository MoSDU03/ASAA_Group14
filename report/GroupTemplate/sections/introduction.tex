\section{Introduction and Motivation}

Modern industrial automation systems face increasing demands for flexibility, reliability, and performance. The beverage manufacturing industry processes millions of units daily, where consistent quality and high throughput are non-negotiable. Traditional time-triggered control architectures provide predictability but lack the flexibility needed for dynamic production environments.

This paper addresses the challenge of designing a control system architecture that balances competing quality attributes: performance (sub-second cycle times), safety (rapid fault detection), and reliability (>99\% successful operations). We focus on the can filling operation, a minimal yet representative scope that demonstrates architectural principles without unnecessary complexity.

\subsection{Problem Statement}

The problem is to architect a control system for automated can filling that meets strict quality requirements while remaining maintainable and extensible. The system must detect can position within ±2mm tolerance, control fill volume to 330ml ±5ml, complete cycles in 600-1500ms, and detect/respond to sensor faults within 200ms. Traditional approaches struggle with the trade-off between loose coupling (for maintainability) and timing predictability (for performance).

\subsection{Research Questions}

This work addresses four key research questions drawn from the course framework:

\begin{enumerate}
    \item \textbf{RQ1}: How can different architectures support the stated system requirements?
    \item \textbf{RQ2}: Which architectural trade-offs must be taken from technology choices?
    \item \textbf{RQ3}: Which parts of architecture design can be modeled, validated, and verified, and what are the results?
    \item \textbf{RQ4}: How can verification results improve architecture design quality?
\end{enumerate}

\subsection{Approach}

We adopt a model-driven methodology based on EAST-ADL~\cite{eastadl_blom}. Our approach consists of five phases:

\textbf{Phase 1}: Requirements elicitation following quality attribute scenario (QAS) templates~\cite{bass_swa}. We defined 15 requirements (8 functional, 7 non-functional) linked to measurable quality attributes.

\textbf{Phase 2}: Architecture design using SysML notation. We created feature models, component diagrams (IBD), state machines, and sequence diagrams following EAST-ADL semantics.

\textbf{Phase 3}: Formal modeling using UPPAAL timed automata. We translated SysML state machines into a network of timed automata with clock constraints matching timing requirements.

\textbf{Phase 4}: Verification using model checking. We verified 15 CTL properties covering deadlock freedom, timing bounds, safety invariants, and liveness properties.

\textbf{Phase 5}: Implementation and empirical validation. We built a Docker-based prototype using Python, MQTT (QoS 1), and PostgreSQL, then validated performance through controlled experiments.

\subsection{Contributions}

This work makes three contributions:

\begin{itemize}
    \item A systematic application of model-driven architecture (MDA) to industrial control, demonstrating how formal methods detect design flaws before implementation.
    \item An event-driven architecture that achieves timing predictability without sacrificing loose coupling, validated through both formal verification and empirical testing.
    \item Empirical evidence showing correlation between formal model predictions and actual system behavior, with mean cycle time of 892ms matching UPPAAL predictions within 4\%.
\end{itemize}

The remainder of this paper is organized as follows: Section II reviews related work on architecture description languages and event-driven systems. Section III presents the use case and quality attribute scenarios. Section IV describes the architecture design. Section V details formal verification with UPPAAL. Section VI presents empirical evaluation results. Section VII concludes with discussion of findings and future work.